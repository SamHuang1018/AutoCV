\documentclass[a4paper,11pt]{article}

%----------------------------------------------------------------------------------------
%	PACKAGES
%----------------------------------------------------------------------------------------
\usepackage{url}
\usepackage{parskip} 	
\RequirePackage{color}
\RequirePackage{graphicx}
\usepackage[usenames,dvipsnames]{xcolor}
\usepackage[scale=0.92]{geometry}
\usepackage{tabularx}
\usepackage{enumitem}
\newcolumntype{C}{>{\centering\arraybackslash}X} 
\usepackage{supertabular}
\newlength{\fullcollw}
\setlength{\fullcollw}{0.47\textwidth}
\usepackage{titlesec}				
\usepackage{multicol}
\usepackage{multirow}

% 中文支援
\usepackage{xeCJK}
\setCJKmainfont{Noto Sans CJK TC}

%CV Sections
\titleformat{\section}{\Large\scshape\raggedright}{}{0em}{}[\titlerule]
\titlespacing{\section}{0pt}{8pt}{8pt}

%Setup hyperref package, and colours for links
\usepackage[unicode, draft=false]{hyperref}
\definecolor{linkcolour}{rgb}{0,0.2,0.6}
\hypersetup{colorlinks,breaklinks,urlcolor=linkcolour,linkcolor=linkcolour}

%for social icons
\usepackage{fontawesome5}

%----------------------------------------------------------------------------------------
%	JOB ENVIRONMENTS
%----------------------------------------------------------------------------------------
\newenvironment{jobshort}[2]
    {
    \begin{tabularx}{\linewidth}{@{}l X r@{}}
    \textbf{#1} & \hfill &  #2 \\[3.75pt]
    \end{tabularx}
    }
    {
    }

\newenvironment{joblong}[2]
    {
    \begin{tabularx}{\linewidth}{@{}l X r@{}}
    \textbf{#1} & \hfill &  #2 \\[3.75pt]
    \end{tabularx}
    \begin{minipage}[t]{\linewidth}
    \begin{itemize}[nosep,after=\strut, leftmargin=1em, itemsep=2pt,label=--]
    }
    {
    \end{itemize}
    \end{minipage}    
    }

%----------------------------------------------------------------------------------------
%	BEGIN DOCUMENT
%----------------------------------------------------------------------------------------
\begin{document}
\pagestyle{empty} 

%----------------------------------------------------------------------------------------
%	HEADER
%----------------------------------------------------------------------------------------
\begin{tabularx}{\linewidth}{@{} C @{}}
\Huge{黃柏翰 (Sam Huang)} \\[7.5pt]
\href{https://github.com/SamHuang1018}{\raisebox{-0.05\height}\faGithub\ SamHuang1018} \ $|$ \ 
\href{https://www.linkedin.com/in/sam-huang-329006244}{\raisebox{-0.05\height}\faLinkedin\ sam-huang} \ $|$ \ 
\href{mailto:ju810609@email.com}{\raisebox{-0.05\height}\faEnvelope \ ju810609@email.com} \ $|$ \ 
\href{tel:0930134819}{\raisebox{-0.05\height}\faMobile \ 0930-134-819} \\
\end{tabularx}

%----------------------------------------------------------------------------------------
%	SUMMARY
%----------------------------------------------------------------------------------------
\section{個人簡介}
機器學習工程師,專精於 LLM/SLM 訓練與部署、語音辨識及生成式 AI 應用。具備金融業地端 AI 解決方案開發經驗,包含智能會議助理、客服話務機器人及文件辨識系統。擅長 GCP 雲端部署與企業級 GPU 資源管理,成功交付多項生產環境 AI 與 ML 系統。

%----------------------------------------------------------------------------------------
%	WORK EXPERIENCE
%----------------------------------------------------------------------------------------
\section{工作經歷}

\begin{joblong}{AI 工程師 - 台新銀行}{2025/07 - 現在}
\item 開發 Breeze-ASR-25 語音辨識 API,取代原有 Whisper 模型,提升銀行應用場景的語音轉文字準確率
\item 建置虛擬人串流系統,設計負載平衡與資源配置機制,實現多人同時使用的客服應用
\item 使用 Flux 模型實作圖片生成 API,新增去背與疊圖功能
\item 開發 Wan 系列模型影片生成流程,將輸出時長從 5 秒延長至 30 秒
\end{joblong}

\begin{joblong}{AI 工程師 - 新光金控}{2024/09 - 2025/06}
\item 部署地端 Whisper 模型取代 Azure STT,批次處理時間減少 84\%,會議記錄製作時間節省 60-70\%
\item 使用 Llama-3.1-8B 開發新光人壽客服話務助理,完成 POC 驗證並獲主管核准
\item 建置醫囑辨識 POC,整合 Gemma-3-27B 與 OCR,辨識準確率達 70-80\%,單張處理時間 <30 秒
\item 建立保單知識庫 RAG 系統,使用 FAISS 索引 60+ 頁保單條款文件
\item 維運法遵 AI 比對模型,使用者回報問題平均 4 天內結案
\end{joblong}

\begin{joblong}{機器學習工程師 - 桑日創意行銷}{2023/12 - 2024/08}
\item 開發 AI 聊天機器人,整合 GPT API 至各通訊平台,使用 RAG 架構與 Function Calling 處理結構化資料查詢
\item 建置語音會議機器人,使用 Fast-Whisper 進行語音轉文字,PyAnnote 進行說話者辨識
\item 實作會議摘要生成、Markmap 心智圖視覺化、Python-pptx 一鍵 PPT 生成功能
\item 使用 Cloud Functions 與 Cloud Run 部署服務,Firebase 作為後端資料庫
\end{joblong}

\begin{joblong}{資料科學家(實習)- 美庫爾}{2023/03 - 2023/09}
\item 為 L'Oréal 建置自動化資料管線,整合 10 個資料源,使用 GCP 服務(Dataflow、Cloud Functions、Airflow、BigQuery)
\item 開發 KFC 顧客分群模型(K-means)與產品偏好預測(SVD++)
\end{joblong}

%----------------------------------------------------------------------------------------
%	SKILLS
%----------------------------------------------------------------------------------------
\section{技術能力}
\begin{tabularx}{\linewidth}{@{}l X@{}}
LLM/SLM &  Llama 3.1、Gemma-3、GPT API、LangChain、RAG、FAISS、Prompt Engineering \\
語音與視覺 &  Whisper、Breeze-ASR、Fast-Whisper、PyAnnote、OCR、Flux、Wan \\
雲端與 MLOps &  GCP(Cloud Functions、Cloud Run、BigQuery、Dataflow)、Azure、Airflow、Docker、Git \\
程式語言 &  Python、SQL、Flask、FastAPI、Streamlit \\
資料庫 &  Firebase、MySQL、Redis、BigQuery \\
\end{tabularx}

%----------------------------------------------------------------------------------------
%	EDUCATION
%----------------------------------------------------------------------------------------
\section{學歷}
\begin{tabularx}{\linewidth}{@{}l X@{}}	
2021 - 2023 & 國立臺北商業大學 財務金融研究所 碩士 \\
& 論文:利用情感分析和交易策略預測金融市場走勢 \\[3pt]
2016 - 2020 & 銘傳大學 財務金融學系 學士 \\
\end{tabularx}

%----------------------------------------------------------------------------------------
%	CERTIFICATIONS
%----------------------------------------------------------------------------------------
\section{證照}
\begin{tabularx}{\linewidth}{@{}l X@{}}
Microsoft &  Azure Data Scientist Associate、Azure AI Engineer Associate \\
Google &  Google Analytics 個人認證 \\
語言 &  TOEIC 700 \\
\end{tabularx}

\vfill
\center{\footnotesize 最後更新:\today}

\end{document}
